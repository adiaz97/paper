\documentclass[a4paper]{article}
\usepackage[spanish]{babel}
\usepackage[utf8]{inputenc}
\usepackage{anysize}
\usepackage{setspace}
\usepackage{amsmath}
\usepackage{hyperref}
\begin{document}
\title{Dinámica del sólido rígido. Giróscopo}
\author{Álvaro Díaz Carmona}
\date{Sesión del 7 de Diciembre de 2016}
\maketitle
\begin{abstract}
\url{https://github.com/adiaz97/paper}

En este informe se reproduce un experimento para analizar el movimiento de un sólido rígido. El objetivo principal es la determinación de uno de los momentos de inercia de este sólido y el estudio de su rotación, precesión y nutación.

\end{abstract}
%
%\section{Introducción}
%\section{Fundamento Teórico}
%Supóngase un sólido libre de girar sin rozamiento en torno a un punto fijo cuya velocidad angular inicial $\omega$ es constante y se encuentra en la parte positiva del eje $Z$. Su momento angular quedará definido por:
%\begin{equation}
%\label{definercia}
%\overrightarrow{L}=I_z \omega \hat{e}_{z}
%\end{equation}
%\section{Material y Metodología}
%\section{Datos recogidos}
%\section{Resultados}
%\section{Conclusiones}
%\section*{Apéndices}
%\subsection*{Apéndice de errores}
%\subsection*{Apéndice de imágenes}
\nocite{*}
\bibliographystyle{unsrt}
\bibliography{mecanica.bib}
\end{document}


